
\documentclass{article}

% to include pdf/eps/png files
\usepackage{graphicx}

% useful to add 'todo' markers
\usepackage{todonotes}

% hyperrefs
\usepackage{hyperref}

% nice source code formatting
\usepackage{minted}

% change style of section headings
\usepackage{sectsty}
\allsectionsfont{\sffamily}

% only required for orgmode ticked TODO items, can remove
\usepackage{amssymb}

% often use this in differential operators:
\renewcommand{\d}{\ensuremath{\mathrm{d}}}

\begin{document}

\title{\sffamily \textbf{Title -- template for orgmode latex production}}

\author{Hans Fangohr, University of Southampton}

\maketitle

\begin{abstract}
  This is an abstract abstract, in the sense of only providing a
  virtual abstract, also known as the interface. Somebody will have to
  provide an inherited class that provides the real abstract.

  We have placed the abstract in the paper.tex file, so that all the
  content in the orgmode file \texttt{content.org} is organised into
  sections, and they can be unfolded, re-arranged, etc (the abstract
  doesn't go well into a section because it appears even before the
  first section starts). If you prefer, you can move the abstract
  latex definition as is into the \texttt{content.org} file, and all
  will work as before.
\end{abstract}


% include body of the paper, auto generated from orgmode content.org file
\input{content.tex}


\bibliographystyle{abbrv}
\bibliography{paper.bib}

\end{document}
